This thesis is comprised of various published research. Chapters are divided as independent papers and provides all necessary introductions and related literature. Here is a brief description of the chapters.

Chapter One provides a brief discussion of Neural Networks and Convolutional Neural Networks that is necessary to follow the subsequent chapters. Only necessary elaboration is provided and user is encouraged to explore various textbooks in the area for a complete and a thorough guide.

Chapter two presents the research which shows how to use Convolutional Neural Network to make image compression that is semantically aware.
It has long been considered a significant problem to improve the visual quality of lossy image and video compression. Recent advances in computing power together with the availability of large training data sets has increased interest in the application of deep learning cnns to address image recognition and image processing tasks. Here, we present a powerful cnn tailored to the specific task of semantic image understanding to achieve higher visual quality in lossy compression. A modest increase in complexity is incorporated to the encoder which allows a standard, off-the-shelf jpeg decoder to be used. While jpeg encoding may be optimized for generic images, the process is ultimately unaware of the specific content of the image to be compressed. Our technique makes jpeg content-aware by designing and training a model to identify multiple semantic regions in a given image. Unlike object detection techniques, our model does not require labeling of object positions and is able to identify objects in a single pass. We present a new cnn architecture directed specifically to image compression, which generates a map that highlights semantically-salient regions so that they can be encoded at higher quality as compared to background regions. By adding a complete set of features for every class, and then taking a threshold over the sum of all feature activations, we generate a map that highlights semantically-salient regions so that they can be encoded at a better quality compared to background regions. Experiments are presented on the Kodak PhotoCD dataset and the MIT Saliency Benchmark dataset, in which our algorithm achieves higher visual quality for the same compressed size.
This chapters presents the work published as --

\noindent\textbf{
Prakash, Aaditya, Nick Moran, Solomon Garber, Antonella DiLillo and James A. Storer. \textit{``Semantic Perceptual Image Compression Using Deep Convolution Networks.''} 2017 Data Compression Conference (DCC - Oral) (2017)
}
\vspace{2em}

Chapter three presents the research which improves upon the results from chapter one by making these images robust in the presence of an adversary.
As deep neural networks (DNNs) have been integrated into critical systems, several methods to attack these systems have been developed. These adversarial attacks make imperceptible modifications to an image that fool DNN classifiers. We present an adaptive JPEG encoder which defends against many of these attacks. Experimentally, we show that our method produces images with high visual quality while greatly reducing the potency of state-of- the-art attacks. Our algorithm requires only a modest increase in encoding time, produces a compressed image which can be decompressed by an off-the-shelf JPEG decoder, and classified by an unmodified classifier.
This chapters presents the work published as --

\noindent\textbf{
Prakash, Aaditya, Nick Moran, Solomon Garber, Antonella DiLillo and James A. Storer. \textit{``Protecting JPEG Images Against Adversarial Attacks.''} 2018 Data Compression Conference (DCC - Oral) (2018)
}
\vspace{2em}

Chapter four extends the work of preventing use of adversarial images from being used to fool the deep networks.
CNNs are poised to become integral parts of many critical systems. Despite their robustness to natural variations, image pixel values can be manipulated, via small, carefully crafted, imperceptible perturbations, to cause a model to misclassify images. We present an algorithm to process an image so that classification accuracy is significantly preserved in the presence of such adversarial manipulations. Image classifiers tend to be robust to natural noise, and adversarial attacks tend to be agnostic to object location. These observations motivate our strategy, which leverages model robustness to defend against adversarial perturbations by forcing the image to match natural image statistics. Our algorithm locally corrupts the image by redistributing pixel values via a process we term pixel deflection. A subsequent wavelet-based denoising operation softens this corruption, as well as some of the adversarial changes. We demonstrate experimentally that the combination of these techniques enables the effective recovery of the true class, against a variety of robust attacks. Our results compare favorably with current state-of-the-art defenses, without requiring retraining or modifying the CNN.
This chapters presents the work published as --

\noindent\textbf{
Prakash, Aaditya, Nick Moran, Solomon Garber, Antonella DiLillo and James A. Storer. \textit{``Deflecting Adversarial Attacks with Pixel Deflection.''} 2018 IEEE/CVF Conference on Computer Vision and Pattern Recognition (CVPR - Spotlight) (2018)
}
\vspace{2em}

Chapter five finds yet another limitation of deep networks - redundancy in feature representation.
A well-trained Convolutional Neural Network can easily be pruned without significant loss of performance. This is because of unnecessary overlap in the features captured by the network's filters. Innovations in network architecture such as skip/dense connections and Inception units have mitigated this problem to some extent, but these improvements come with increased computation and memory requirements at run-time. We attempt to address this problem from another angle - not by changing the network structure but by altering the training method. We show that by temporarily pruning and then restoring a subset of the model's filters, and repeating this process cyclically, overlap in the learned features is reduced, producing improved generalization. We show that the existing model-pruning criteria are not optimal for selecting filters to prune in this context and introduce inter-filter orthogonality as the ranking criteria to determine under-expressive filters. Our method is applicable both to vanilla convolutional networks and more complex modern architectures, and improves the performance across a variety of tasks, especially when applied to smaller networks.
This chapters presents the work published as --

\noindent\textbf{
Prakash, Aaditya, James A. Storer, Dinei A. F. Florêncio and Cha Zhang. \textit{``RePr: Improved Training of Convolutional Filters.''} 2019 IEEE/CVF Conference on Computer Vision and Pattern Recognition (CVPR - Oral) (2019)
}
\vspace{2em}

Chapter six gives a brief description of how to extend Convolutional Neural Networks to incorporate languages when the task involves multimodal signals.
We propose a version of highway network designed for the task of Visual Question Answering. We take inspiration from recent success of Residual Layer Network and Highway Network in learning deep representation of images and fine grained localization of objects. We propose variation in gating mechanism to allow incorporation of word embedding in the information highway. The gate parameters are influenced by the words in the question, which steers the network towards localized feature learning. This achieves the same effect as soft attention via recurrence but allows for faster training using optimized feed-forward techniques. We are able to obtain state-of-the-art1 results on VQA dataset for Open Ended and Multiple Choice tasks with current model.
This chapters presents the work published as --

\noindent\textbf{
Prakash, Aaditya and James Storer Brandeis. \textit{``Highway Networks for Visual Question Answering.''} IEEE/CVF Conference on Computer Vision and Pattern Recognition (VQA - Workshop - Spotlight) (2016)
}
\vspace{2em}

Chapter seven goes deeper into languge domains and explores the idea of residual connection in the context of languages.
we propose a novel neural approach for paraphrase generation. Conventional paraphrase generation methods either leverage handwritten rules and thesauri-based alignments, or use statistical machine learning principles. To the best of our knowledge, this work is the first to explore deep learning models for paraphrase generation. Our primary contribution is a stacked residual LSTM network, where we add residual connections between LSTM layers. This allows for efficient training of deep LSTMs. We experiment with our model and other state-of-the-art deep learning models on three different datasets: PPDB, WikiAnswers and MSCOCO. Evaluation results demonstrate that our model outperforms sequence to sequence, attention-based and bi-directional LSTM models on BLEU, METEOR, TER and an embedding-based sentence similarity metric.
This chapters presents the work published as --

\noindent\textbf{
Prakash, Aaditya, Sadid A. Hasan, Kathy Lee, Vivek Datla, Ashequl Qadir, Joey Liu and Oladimeji Farri. \textit{``Neural Paraphrase Generation with Stacked Residual LSTM Networks.''} COLING (2016) 
}
\vspace{2em}


Chapter eight explores memory networks and shows its efficacy in diagnosis of diseses.
Diagnosis of a clinical condition is a challenging task, which often requires significant medical investigation. Previous work related to diagnostic inferencing problems mostly consider multivariate observational data (e.g. physiological signals , lab tests etc.). In contrast, we explore the problem using free-text medical notes recorded in an electronic health record (EHR). Complex tasks like these can benefit from structured knowledge bases, but those are not scalable. We instead exploit raw text from Wikipedia as a knowledge source. Memory networks have been demonstrated to be effective in tasks which require comprehension of free-form text. They use the final iteration of the learned representation to predict probable classes. We introduce condensed memory neural networks (C-MemNNs), a novel model with iterative condensation of memory representations that preserves the hierarchy of features in the memory. Experiments on the MIMIC-III dataset show that the proposed model outperforms other variants of memory networks to predict the most probable diagnoses given a complex clinical scenario.
This chapters presents the work published as --

\noindent\textbf{
Prakash, Aaditya, Siyuan Zhao, Sadid A. Hasan, Vivek Datla, Kathy Lee, Ashequl Qadir, Joey Liu and Oladimeji Farri. \textit{``Condensed Memory Networks for Clinical Diagnostic Inferencing.''} AAAI (2017)
}
\vspace{2em}
